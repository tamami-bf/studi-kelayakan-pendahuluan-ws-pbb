\documentclass[pdftex, 12pt, oneside]{article}

%\usepackage[paperwidth=8.5in, paperheight=13in]{geometry} % folio
\usepackage[paperwidth=8.27in, paperheight=11.69in]{geometry} % A4
\usepackage{makeidx} % allow index generation
\usepackage{graphicx} % standard latex graphics tool when including figure files
\usepackage[bottom]{footmisc} % places footnotes at page bottom
\usepackage[english]{babel}
\usepackage{enumerate}
\usepackage{paralist}
\usepackage{float}
\usepackage{gensymb}
\usepackage{listings}

\renewcommand{\baselinestretch}{1.5}

\newcommand{\HRule}{\rule{\linewidth}{0.5mm}}

\begin{document}
\sloppy

\begin{center}
{\large STUDI KELAYAKAN PENDAHULUAN PEMBANGUNAN LAYANAN \textit{WEB SERVICES} SEBAGAI CARA KOMUNIKASI DALAM PENCATATAN PEMBAYARAN PAJAK BUMI DAN BANGUNAN PERDESAAN DAN PERKOTAAN DI KABUPATEN BREBES}\\[1cm]
26 Agustus 2016\\
Priyanto Tamami, S.Kom.
\end{center}

\section{PENENTUAN MASALAH DAN PELUANG YANG DITUJU SISTEM}
                                               
\subsection{Penentuan Masalah}           

Pada sejarah bangsa Indonesia, pajak atas bumi dan bangunan dapat dikatakan pajak yang paling tua. Pajak Bumi dan Bangunan (PBB) sejatinya sudah ada sejak masa sebelum penjajahan hingga saat ini, hanya saja aturan perpajakan yang diterapkan berbeda-beda pada masing-masing zaman. 

Dahulu, rakyat Indonesia sudah dibebani dengan persembahan upeti dalam bentuk natura kepada para penguasa sebagai tanda pengakuan atas kepemimpinan dan bukti rasa syukur atas pengayoman dari penguasa tersebut. Pada masa penjajahan Belanda, pajak bumi dikenal dengan nama \textit{Land Rent}. Ketentuannya saat itu bahwa \textit{Land Rent} dikenakan terhadap semua jenis tanah produktif dan wajib pajaknya adalah desa (Kepala Desa) dan bukan perseorangan, karena kepala desa dianggap sebagai penyewa yang harus membayar sewa tanah. Besarnya tarif Land Rent bervariasi antara 20\% hingga 50\% dari hasil produksi pertanian tergantung pada jenis produksinya.

Pada masa penjajahan Jepang, Land Rent atau Landrente diubah menjadi Land Tax. Administrasi Pajak ditangani oleh kantor pajak yang disebut \textit{Zaimubu Shuzeika} yang sekaligus bertugas untuk melakukan survey dan pemetaan di pulai Jawa dan Madura.

Pada masa setelah kemerdekaan, pemungutan pajak jenis ini masih berlangsung dengan nama Pajak Bumi yang kemudian diganti dengan Pajak Pendapatan Tanah. Periode tahun 1945 sampai tahun 1951. Kemudian dengan desakan dari golongan yang dipimpin oleh Tauchid, yang dikenal dengan nama Mosi Tauchid, maka Pajak Pendapatan Tanah pun dihapuskan, sebagai gantinya, dikeluarkan pajak baru dengan nama Pajak Penghasilan atas Tanah Pertanian (PPTP).

Pada tahun 1951 sampai tahun 1959, setelah dikeluarkan UU Nomor 14 Tahun 1951 tentang Penghapusan Pajak Bumi di wilayah Negara Republik Indonesia, maka lahirlah Jawatan Pendaftaran dan Pajak Penghasilan Tanah Milik Indonesia (P3TMI) yang bertugas melakukan pendaftaran atas tanah-tanah milik adat yang ada di Indonesia. Karena tugasnya hanya mengurus pendaftaran tanah saja, maka namanya diubah kembali menjadi jawatan Pendaftaran Tanah Milik Indonesia (PTMI) dan bertugas sama seperti sebelumnya ditambah dengan kewenangan untuk mengeluarkan Surat Pendaftaran sementara terhadap tanah milik yang sudah terdaftar.

Pada tahun 1959 sampai tahun 1985, nama jawatan yang mengelola Pajak Hasil Bumi menjadi Direktorat Pajak Hasil Bumi yang dalam pelaksanaannya diubah namanya menjadi Direktorat Iuran Pembangunan Daerah (DIT-IPEDA), dan nama Pajak Hasil Bumi diubah menjadi Iuran Pembangunan Daerah (IPEDA). Pengenaannya diberlakukan pada tanah-tanah sektor perdesaan, perkotaan, perhutanan, perkebunan dan pertambangan.

Selain IPEDA, pada masa itu dipungut pula 6 (enam) pajak kekayaan dan pungutan lain atas tanah dan bangunan yang menimbulkan tumpang tindih antara satu pajak dengan pajak lainnya dan menyebabkan adanya beban pajak berganda bagi masyarakat. Dengan adanya reformasi perpajakan pertama yang dimulai pada tahun 1983, antara lain dengan penyederhanaan jumlah dan jenis pajak atas tanah dan bangunan melalui pengundangan UU Nomor 12 Tahun 1985, maka 7 (tujuh) jenis pajak kebendaan dan kekayaan atas tanah dan bangunan disederhanakan menjadi PBB.

Pemberlakuan UU Nomor 12 Tahun 1985 tentang Pajak Bumi dan Bangunan didasari pemikiran antara lain bahwa bumi dan bangunan memberikan keuntungan dan atau kedudukan sosial ekonomi yang lebih baik bagi orang atau badan yang mempunyai suatu hak atasnya dan memperoleh manfaat darinya, oleh sebab itu wajar apabila kepada mereka diwajibkan memberikan sebagian dari manfaat atau kenikmatan yang diperolehnya kepada negara melalui pajak.

Pelaksanaan reformasi di bidang pajak atas tanah dan bangunan disamping berupaya menyederhanakan berbagai pungutan pajak atas tanah dan bangunan juga tetap memberikan tekanan terhadap upaya untuk meningkatkan penerimaan dan memperhatikan aspek keadilan serta meminimalkan dampak terhadap distorsi kegiatan ekonomi dan sosial mengingat PBB merupakan salah satu sumber utama penerimaan daerah mengingat PBB adalah penerimaan pajak Pusat yang keseluruhan hasilnya diserahkan kepada Daerah.

Menyadari pentingnya penerimaan PBB bagi pembiayaan pembangunan Daerah, maka pada tahun 1989 dilakukan pembaharuan sistem administrasi penerimaan PBB melalui Sistem Tempat Pembayaran (SISTEP). SISTEP diujicobakan pertama kali pada tahun 1989 di wilayah Kabupaten Tangerang, yang secara bertahap sehingga pada akhir tahun 1994 seluruh Kabupaten/Kota di Indonesia telah melaksanakan sistem administrasi pemungutan PBB dengan pola SISTEP.

Pokok-pokok ketentuan SISTEP antara lain meliputi :

\begin{itemize}
  \item Hanya ada satu tempat pembayaran untuk setiap wilayah pembayaran PBB tertentu sebagaimana tercantum dalam Surat Pemberitahuan Pajak Terhutang (SPPT) yang diusahakan berdekatan dengan lokasi objek pajak.
  \item Pembayaran PBB dilakukan sekaligus dalam satu kali pembayaran dan tidak dapat diangsur.
  \item Jatuh tempo pembayaran PBB diatur seragam sehingga hanya terdapat satu tanggal jatuh tempo.
  \item Surat Tanda Terima Setoran (STTS) PBB telah tersedia di tempat pembayaran sebelum SPPT diterima oleh wajib pajak.
  \item Administrasi PBB harus dilaksanakan dengan dukungan komputer.
  \item Sistem Pemantauan STTS dan pelaporan pembayaran didesain sedemikian rupa sehingga perkembangan pembayaran PBB diketahui lebih cepat oleh instansi terkait.
  \item Secara sistem, SISTEP mampu menerbitkan daftar negatif (\textit{negative list}) wajib pajak yang tidak memenuhi kewajiban PBB pada saat jatuh tempo pembayaran sehingga penegakan hukum dapat dilaksanakan.
\end{itemize}

Dengan diberlakukannya SISTEP, penerimaan PBB mengalami peningkatan yang berarti. Keberhasilan pembaharuan sistem administrasi pemungutan dengan pola SISTEP terutama menyangkut perubahan sistem pemungutan PBB yang sebelumnya dilakukan oleh petugas pemungut desa/kelurahan secara bertahap diambil alih melalui sistem perbankan yang ditunjang dengan komputerisasi administrasi penerimaan PBB. Pemberian pelayanan kepada wajib pajak dalam rangka pembayaran PBB menjadi lebih mudah dan pelaksanaan penegakan hukum dapat lebih ditingkatkan.

Pada tahun 1994 Pemerintah menerbitkan UU Nomor 12 Tahun 1994 tentang Perubahan UU Nomor 12 Tahun 1985 tentang Pajak Bumi dan Bangunan, esensi dari perubahan ini adalah penyesuaian atas praktek perkembangan penyelenggaraan kegiatan usaha yang belum tertampung dalam UU Nomor 12 Tahun 1985. Pada saat inilah Sistem Manajemen Informasi Objek Pajak yang dikenal dengan nama SISMIOP muncul dengan perbaikan pada manajemen objek pajak berupa pemberian Nomor Objek Pajak (NOP) yang baku.

Selanjutnya seiring dengan bergulirnya reformasi di Indonesia, dan sejalan dengan tuntutan otonomi daerah setelah masa reformasi maka beberapa Pemda menuntut agar PBB P2 (Perdesaan dan Perkotaan) menjadi pajak daerah, namun tidak semua Pemda setuju sehingga terjadi perdebatan pro dan kontra pengalihan PBB P2 menjadi pajak daerah. Pemda yang setuju adalah mereka yang memiliki potensi penerimaan PBB P2 yang besar (kota-kota besar di Indonesia) karena akan menambah jumlah PAD-nya, jika masih menjadi pajak pusat maka PBB P2 yang kembali ke daerah penghasil dalam bentuk Dana Bagi Hasil (DBH) hanya 64,8\% saja. Sementara itu Pemda yang tidak setuju dengan peralihan PBB P2 menjadi pajak daerah adalah Pemda yang memiliki potensi penerimaan PBB P2 yang kecil (Daerah yang bukan kota besar di Indonesia), selain itu dikeluhkan juga faktor lain diluar pendapatan seperti :

\begin{itemize}
  \item Sarana dan prasarana yang mendukung
  \item Keterbatasan SDM yang mampu mengelola PBB P2
  \item Regulasi dan SOP Pelayanan
  \item Sistem Manajemen Informasi Objek Pajak
\end{itemize}

Selain itu tantangan pengalihan PBB P2 bagi Pemda antara lain kesiapan Kabupaten/Kota pada masa awal pengalihan yang belum optimal, sehingga dapat berdampak pada penurunan pelayanan, penerimaan, dan beberapa hal seperti :

\begin{itemize}
  \item Kesenjangan (disparitas) kebijakan PBB P2 antar Kabupaten/Kota.
  \item Hilangnya potensi penerimaan bagi Provinsi (16,2\%) dan hilangnya potensi penerimaan insentif PBB khususnya bagi Kabupaten/Kota yang potensi PBB P2-nya rendah.
  \item Beban biaya pemungutan PBB-P2 yang cukup besar.
\end{itemize}

Kemudian pada tanggal 15 September 2009 terbitlah UU Nomor 28 Tahun 2009 tentang Pajak Daerah dan Retribusi Daerah dimana telah mengakomodir PBB P2 menjadi pajak daerah yang paling lambat dilaksanakan pada tanggal 1 Januari 2014.

Adapun tujuan dari pengalihan PBB-P2 menjadi pajak daerah adalah untuk meningkatkan \textit{local taxing power} pada Kabupaten/Kota seperti : 

\begin{itemize}
  \item Memperluas objek pajak daerah dan retribusi daerah
  \item Menambah jenis pajak daerah dan retribusi daerah (termasuk pengalihan PBB Perdesaan dan Perkotaan di Pajak Daerah)
  \item Memberikan diskresi penetapan tarif pajak kepada daerah
  \item Menyerahkan fungsi pajak sebagai instrumen penganggaran dan pengaturan pada daerah.
\end{itemize}

Pada awal pengalihannya, kondisi pembayaran PBB P2 di Kabupaten Brebes menggunakan sistem \textit{single host} dimana data ditempatkan dalam 2 (dua) tempat, yaitu di Dinas Pendapatan dan Pengelolaan (DPPK) sebagai Dinas yang mengelola PBB P2, dan Bank Pembangunan Daerah (BPD) Jawa Tengah sebagai tempat pembayaran. 

Sistem ini sangat baik dan efektif untuk melakukan pencatatan pembayaran dengan karakteristik Kabupaten Brebes yang memiliki jumlah objek pajak lebih dari 1 (satu) juta, yang pola pembayarannya mayoritas dilakukan oleh petugas pemungut ditingkat desa/kelurahan, sehingga dalam 1 (satu) hari bisa terjadi puluhan, ratusan, hingga ribuan objek pajak yang dibayarkan sekaligus.

Kondisi data pada sistem ini harus selalu diselaraskan dalam periode harian, sehingga data pembayaran yang masuk hari ini akan tercatat dalam basis data DPPK pada H+1, begitupun segala perubahan yang terjadi di basis data DPPK pada hari ini akan dicatatkan perubahannya pada basis data BPD Jawa Tengah di H+1.

Kegiatan ini menjadikan data tidak konsisten dan terjadi perbedaan data diantara 2 (dua) basis data.

\subsection{Peluang Yang Dituju}

Dari kondisi data yang tidak konsisten seperti dijelaskan diatas. Maka diperlukan sebuah sistem yang dapat menjaga konsistensi data sekaligus mampu melakukan tugasnya mencatat transaksi pembayaran massal dan perubahan data ketetapan setiap harinya.

Salah satu solusi yang dapat diambil adalah dengan membangun sistem Host-to-Host dengan seluruh data berada di DPPK sebagai pengelola PBB P2, sedangkan tempat pembayaran akan meminta dan mencatatkan data sesuai dengan objek yang akan dibayarkan saja langsung ke basis data yang berada di DPPK.
                              
\section{PEMBENTUKAN SASARAN SISTEM BARU SECARA KESELURUHAN}

Sasaran dari sistem baru secara keseluruhan adalah tentunya dapat mencatatkan data pembayaran massal secara cepat, hal ini karena hanya sebagian kecil saja dari wajib pajak yang melakukan pembayaran PBB-P2 langsung ke tempat pembayaran, sebagian besar wajib pajak melakukan pembayaran melalui petugas pemungut Desa/Kelurahan, yang dalam periode tertentu, mungkin mingguan, atau bulanan, petugas pemungut akan menyetorkannya ke tempat pembayaran, sehingga yang terjadi adalah dalam satu kali transaksi penyetoran ke tempat pembayaran, bisa dimungkinkan akan ada lebih dari 100 objek sekaligus yang terjadi di hari itu.

Sasaran lain dari sistem baru ini yaitu terjaganya konsistensi data, basis data cukup berada di DPPK Kabupaten Brebes sebagai pengelola PBB-P2, apabila ada transaksi pembayaran, BPD Jawa Tengah selaku tempat pembayaran akan mencatatkannya langsung ke basis data PBB di DPPK, termasuk perubahan-perubahan yang terjadi dalam basis data bila ada pengajuan pelayanan oleh wajib pajak maka BPD Jawa Tengah dapat mengaksesnya untuk kemudian wajib pajak / petugas pemungut mendapatkan informasi tagihan terbaru.

Karena sistem datanya terpusat, dan pencatatan data pembayarannya dilakukan di dalam satu basis data, maka akan didapatkan data pembayaran yang \textit{realtime}, dimana realisasi penerimaan PBB-P2 akan dapat disajikan angkanya sampai dengan detik pada saat data diakses.


\section{PENGIDENTIFIKASIAN PARA PEMAKAI SISTEM}

Sistem yang akan dibangun ini dapat dikatakan berbentuk layanan, yaitu layanan \textit{inquiry}, atau permintaan informasi data, dalam hal ini adalah data tagihan PBB-P2; layanan pencatatan transaksi pembayaran; dan layanan reversal, atau pengembalian/pembatalan data transaksi pembayaran yang telah dilakukan prosesnya karena beberapa hal.

Sebagai sebuah layanan yang memberikan respon cepat dalam memproses banyak data, maka respon dari sistem ini berbentuk kode yang terkadang sulit dipahami oleh pengguna komputer awam. Maka dari itu, pemakai yang memungkinkan dari sistem ini adalah sebuah sistem juga di pihak Bank sebagai tempat pembayaran, yang nantinya dapat menerjemahkan kode-kode yang diberikan sistem ini sebagai sebuah respon ke dalam tampilan yang mudah dimengerti dan dipahami oleh pengguna komputer awam.


\section{PEMBENTUKAN LINGKUP SISTEM}

Karena luasnya pengertian layanan \textit{web services} ini, maka perlu dikerucutkan, atau dibatasi yaitu hanya akan membahas layanan \textit{web services} menggunakan metode \textit{Rest}, dan layanan \textit{web services}-nya pun hanya akan mengakomodir 3 (tiga) bentuk permintaan saja, yaitu \textit{inquiry}, pencatatan pembayaran, dan reversal.


\section{PENGUSULAN PERANGKAT LUNAK DAN PERANGKAT KERAS UNTUK SISTEM BARU}

Ada beberapa hal yang perlu dipertimbangkan terhadap pengusulan perangkat lunak untuk sistem baru yang akan dibangun, yaitu :

\begin{itemize}
  \item Basis data
  
  Karena data yang digunakan untuk kepentingan transaksi bersifat \textit{realtime}, segala perubahan data yang terjadi pada detik itu juga harus dapat diakses, maka kebutuhan akan basis data baru tidak diperlukan, karena nantinya sistem aplikasi akan melakukan akses langsung terhadap basis data Sistem Manajemen Informasi Objek Pajak (SISMIOP) PBB-P2 yang masih menggunakan sistem basis data Oracle 11g. Hanya saja nantinya akan ada penambahan tabel yang diperlukan guna melakukan pencatatan kejadian pada saat sistem aplikasi melakukan transaksi, sehingga memudahkan melakukan pelacakan suatu kejadian apabila terjadi hal-hal yang tidak diinginkan.
  
  \item \textit{Text Editor}
  
  Karena membangun layanan \textit{web services} yang terdiri dari 3 (tiga) layanan saja cukup mudah, maka hanya diperlukan \textit{text editor} sebagai media menuliskan kode programnya. Kali ini menggunakan \textit{text editor} Atom yang tersedia secara gratis.
  
  \item \textit{Pustaka (\textit{Library})}
  
  Untuk mempercepat dan mempermudah membangun sebuah layanan \textit{web services}, maka diperlukan beberapa pustaka atau \textit{library}, \textit{library} yang digunakan untuk membangun layanan \textit{web services} ini yaitu :
  
  \begin{itemize}
    \item Spring. \textit{Framework} yang digunakan untuk mempermudah membangun layanan \textit{web services}
    \item Hibernate, adalah \textit{framework} yang bertugas untuk melakukan komunikasi data antara sistem aplikasi dengan sistem basis data.
    \item BoneCP, adalah pustaka yang membantu \textit{framework} Hibernate dalam melakukan koneksi dan transaksi secara cepat dengan sistem basis data.
    \item Jackson, yaitu pustaka yang membantun mengubah objek Java menjadi format JSON yang nantinya JSON inilah format yang dikirim sebagai respon dari sebuah \textit{request} dari tempat pembayaran, dimana format ini cukup ringan bila digunakan sebagai cara komunikasi data, karena hanya berupa teks, sehingga mampu menjadikan komunikasi dengan sistem tempat pembayaran lebih efisien.
  \end{itemize}
  
  Seluruh pustaka tersebut dapat diunduh dan digunakan secara gratis.
  
  \item \textit{Servlet Container}
  
  Karena sistem yang akan dibangun berbasis Java, maka diperlukan \textit{Servlet Container} sebagai \textit{server}-nya. Untuk kebutuhan ini digunakan Tomcat \textit{server}. \textit{Server} ini dapat diunduh dan digunakan secara gratis.
  
\end{itemize}

Dan kebutuhan akan perangkat keras untuk sistem baru yang akan dibangun, yaitu :

\begin{itemize}
  \item Server Basis Data
  
  Server basis data diperlukan untuk menyimpan data-data hasil transaksi yang terjadi, termasuk didalamnya adalah menyediakan data-data yang diminta oleh tempat pembayaran sebagai informasi berapa besar nilai PBB-P2 yang terhutang. Server basis data ini tidak perlu ditambah lagi, cukup menggunakan server basis data yang sedang berjalan untuk kebutuhan produksi, karena sistem aplikasi yang akan dibangun nantinya akan memanfaatkan /menggunakan data atau mencatatkan data pembayaran ke server basis data ini.
  
  \item Server Aplikasi
  
  Server aplikasi diperlukan sebagai tempat sistem aplikasi dipasangkan. Kedepannya diharapkan server aplikasi ini mampu menangani lebih dari ratusan transaksi setiap harinya. Dari server aplikasi ini pula, sistem aplikasi melakukan komunikasi dengan server basis data dalam melakukan \textit{inquiry} data atau pencatatan pembayaran.
  
  \item \textit{Router}
  
  \textit{Router} diperlukan sebagai penghubung antara jaringan internet dan jaringan intranet. Karena server aplikasi layanan \textit{web services} berada pada jaringan intranet, dan komunikasi yang terjadi dengan tempat pembayaran dilakukan melalui jaringan internet, maka sudah seharusnya menggunakan \textit{Router} untuk menghubungkan kedua jaringan yang berbeda ini.
  
  \item VPN Server
  
  VPN Server adalah \textit{server} yang menyediakan layanan \textit{Virtual Private Network}, nantinya, \textit{client} yang terhubung dengan VPN Server ini akan memiliki alamat IP yang berada dalam satu jaringan dengan jaringan lokal. Sehingga \textit{client} yang terhubung dengan VPN Server akan dianggap seperti mengakses dari jaringan intranet, walaupun sebenarnya \textit{client} yang terhubung, mengakses melalui jaringan internet yang secara geografis terletak sangat jauh.
  
  \item Akses Internet
  
  Tentunya kebutuhan yang tidak kalah penting adalah akses internet yang cukup. Cukup yang dimaksud disini adalah kondisi lebar pita (\textit{bandwidth}) yang digunakan mampu menangani transaksi yang terjadi terutama pada hari-hari dimana transaksi yang terjadi mencapai lebih dari 100 transaksi. 
  
  Namun bukan hanya cukup dalam kondisi lebar pita, tetapi juga kondisi \textit{up-time} atau ketersediaan jaringan, jangan sampai pada saat transaksi sedang pada puncaknya, kemudian koneksi internet terputus karena sedang ada perbaikan jaringan, maka perlu adanya 2 (dua) atau lebih koneksi internet dari penyedia layanan internet yang berbeda, yang diharapkan, ada jaringan yang berfungsi sebagai jaringan internet \textit{standby} yang apabila kondisi jaringan internet primer sedang mengalami gangguan, maka tempat pembayaran dapat langsung melakukan akses ke jaringan internet yang lain yang tersedia.
\end{itemize}


\section{PEMBUATAN ANALISIS UNTUK MEMBANGUN SENDIRI ATAU MEMBELI APLIKASI}

Keputusan untuk membangun aplikasi sendiri atau membeli tentunya akan berdasarkan kepada waktu, kualitas, dan layanan.

Dari sisi waktu, tentunya perusahaan \textit{switching} yang menyediakan layanan \textit{Host-to-Host} dengan bank akan lebih cepat pengerjaannya karena fokus pengerjaan terarah dengan baik, apalagi bila perusahaan tersebut telah melakukan kegiatan dibidang yang sama cukup lama, artinya akan mendapatkan pengalaman permasalahan yang mungkin muncul pada saat kegiatan operasional berjalan. Dengan membangunnya sendiri, tentu dibutuhkan waktu yang cukup lama karena perlu waktu untuk pembelajaran terhadap model \textit{web services} terlebih dahulu, bagaimana teknis \textit{server web services} menerima data, bagaimana data diolah, dan bagaimana \textit{server} harus merespon data terhadap \textit{request} yang datang. Kemudian perlu waktu untuk pembelajaran dan implementasi sistem jaringan yang cukup aman untuk berkomunikasi antara \textit{server web services} dengan tempat pembayaran.

Dari sisi kualitas, tentunya bisa dilihat dari banyaknya \textit{bugs} atau permasalahan kode program yang muncul. Beberapa perusahan yang telah bergerak di bidang ini cukup lama biasanya akan memiliki \textit{bugs} yang sangat sedikit, yang tentunya ini didapatkan dari pengalaman dalam membangun dan memelihara sistem \textit{switching} dengan tempat pembayaran. Dengan kondisi demikian, maka dengan membangun sendiri sistem ini, tentunya akan mengalami beberapa kendala dengan kualitas aplikasi yang dihasilkan, namun akan menjadi investasi jangka panjang bagi proses pemeliharaan sistem aplikasi dan pembaruannya tanpa harus menambahkan biaya tambahan kembali.

Sedangkan dari sisi layanan, seperti hal yang biasa diajukan oleh sebuah \textit{software house}, maka ada 2 (dua) model dasar yang biasanya ditawarkan, yaitu model kontrak putus, artinya begitu sistem sudah jadi dan terpasang, akan ditunjuk 1 (satu) atau lebih orang untuk melakukan \textit{maintenance} dasar, atau bahkan akan diserahkan pula kode sumbernya bila suatu saat akan dilakukan pembaruan atau perbaikan secara mandiri. Model yang kedua yaitu sistem kontrak layanan jasa, yang biasanya sistem aplikasi akan diberikan secara gratis, namun tidak diberikan kode sumbernya, melainkan akan dilakukan perbaikan atau pembaruan secara berkala bila diketahui ada permasalahan pada sistem yang berjalan. Dari model kedua ini pun biasanya akan diberikan sebuah \textit{space server} sebagai sumber data cadangan yang apabila \textit{server} DPPK mengalami masalah baik dari perangkat keras \textit{server} itu sendiri, atau mungkin ada masalah dengan jaringan yang menghubungkan dengan tempat pembayaran, maka \textit{space server} yang dimiliki oleh penyedia jasa akan berperan sebagai penyedia data sehingga ketersediaan data akan dapat diakses oleh tempat pembayaran dalam 99,9%.

Dengan membangun sendiri sistem aplikasi ini, maka tentunya layanan penuh akan dikendalikan oleh personil yang menangani bidang teknologi informasi tanpa tambahan biaya apapun selain dari menggunakan 2 (dua) jaringan internet atau lebih sebagai strategi untuk menjaga konektivitas dengan tempat pembayaran, atau menggunakan layanan \textit{cloud} dengan maksud menjaga konektivitas juga dengan tempat pembayaran.


\section{PEMBUATAN ANALISIS BIAYA / MANFAAT}


\section{PENGKAJIAN TERHADAP RESIKO PROYEK}


\section{PEMBERIAN REKOMENDASI UNTUK MENERUSKAN ATAU MENGHENTIKAN PROYEK}


\end{document}  