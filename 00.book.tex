\documentclass[pdftex, 12pt, oneside]{article}

%\usepackage[paperwidth=8.5in, paperheight=13in]{geometry} % folio
\usepackage[paperwidth=8.27in, paperheight=11.69in]{geometry} % A4
\usepackage{makeidx} % allow index generation
\usepackage{graphicx} % standard latex graphics tool when including figure files
\usepackage[bottom]{footmisc} % places footnotes at page bottom
\usepackage[english]{babel}
\usepackage{enumerate}
\usepackage{paralist}
\usepackage{float}
\usepackage{gensymb}
\usepackage{listings}

\renewcommand{\baselinestretch}{1.5}

\newcommand{\HRule}{\rule{\linewidth}{0.5mm}}

\begin{document}
\sloppy

\begin{center}
{\large STUDI KELAYAKAN PENDAHULUAN PEMBANGUNAN LAYANAN \textit{WEB SERVICES} SEBAGAI CARA KOMUNIKASI DALAM PENCATATAN PEMBAYARAN PAJAK BUMI DAN BANGUNAN PERDESAAN DAN PERKOTAAN DI KABUPATEN BREBES}\\[1cm]
26 Agustus 2016\\
Priyanto Tamami, S.Kom.
\end{center}

\section{PENENTUAN MASALAH DAN PELUANG YANG DITUJU SISTEM}
                                               
\subsection{Penentuan Masalah}           

Pada sejarah bangsa Indonesia, pajak atas bumi dan bangunan dapat dikatakan pajak yang paling tua. Pajak Bumi dan Bangunan (PBB) sejatinya sudah ada sejak masa sebelum penjajahan hingga saat ini, hanya saja aturan perpajakan yang diterapkan berbeda-beda pada masing-masing zaman. 

Dahulu, rakyat Indonesia sudah dibebani dengan persembahan upeti dalam bentuk natura kepada para penguasa sebagai tanda pengakuan atas kepemimpinan dan bukti rasa syukur atas pengayoman dari penguasa tersebut. Pada masa penjajahan Belanda, pajak bumi dikenal dengan nama \textit{Land Rent}. Ketentuannya saat itu bahwa \textit{Land Rent} dikenakan terhadap semua jenis tanah produktif dan wajib pajaknya adalah desa (Kepala Desa) dan bukan perseorangan, karena kepala desa dianggap sebagai penyewa yang harus membayar sewa tanah. Besarnya tarif Land Rent bervariasi antara 20\% hingga 50\% dari hasil produksi pertanian tergantung pada jenis produksinya.

Pada masa penjajahan Jepang, Land Rent atau Landrente diubah menjadi Land Tax. Administrasi Pajak ditangani oleh kantor pajak yang disebut \textit{Zaimubu Shuzeika} yang sekaligus bertugas untuk melakukan survey dan pemetaan di pulai Jawa dan Madura.

Pada masa setelah kemerdekaan, pemungutan pajak jenis ini masih berlangsung dengan nama Pajak Bumi yang kemudian diganti dengan Pajak Pendapatan Tanah. Periode tahun 1945 sampai tahun 1951. Kemudian dengan desakan dari golongan yang dipimpin oleh Tauchid, yang dikenal dengan nama Mosi Tauchid, maka Pajak Pendapatan Tanah pun dihapuskan, sebagai gantinya, dikeluarkan pajak baru dengan nama Pajak Penghasilan atas Tanah Pertanian (PPTP).

Pada tahun 1951 sampai tahun 1959, setelah dikeluarkan UU Nomor 14 Tahun 1951 tentang Penghapusan Pajak Bumi di wilayah Negara Republik Indonesia, maka lahirlah Jawatan Pendaftaran dan Pajak Penghasilan Tanah Milik Indonesia (P3TMI) yang bertugas melakukan pendaftaran atas tanah-tanah milik adat yang ada di Indonesia. Karena tugasnya hanya mengurus pendaftaran tanah saja, maka namanya diubah kembali menjadi jawatan Pendaftaran Tanah Milik Indonesia (PTMI) dan bertugas sama seperti sebelumnya ditambah dengan kewenangan untuk mengeluarkan Surat Pendaftaran sementara terhadap tanah milik yang sudah terdaftar.

Pada tahun 1959 sampai tahun 1985, nama jawatan yang mengelola Pajak Hasil Bumi menjadi Direktorat Pajak Hasil Bumi yang dalam pelaksanaannya diubah namanya menjadi Direktorat Iuran Pembangunan Daerah (DIT-IPEDA), dan nama Pajak Hasil Bumi diubah menjadi Iuran Pembangunan Daerah (IPEDA). Pengenaannya diberlakukan pada tanah-tanah sektor perdesaan, perkotaan, perhutanan, perkebunan dan pertambangan.

Selain IPEDA, pada masa itu dipungut pula 6 (enam) pajak kekayaan dan pungutan lain atas tanah dan bangunan yang menimbulkan tumpang tindih antara satu pajak dengan pajak lainnya dan menyebabkan adanya beban pajak berganda bagi masyarakat. Dengan adanya reformasi perpajakan pertama yang dimulai pada tahun 1983, antara lain dengan penyederhanaan jumlah dan jenis pajak atas tanah dan bangunan melalui pengundangan UU Nomor 12 Tahun 1985, maka 7 (tujuh) jenis pajak kebendaan dan kekayaan atas tanah dan bangunan disederhanakan menjadi PBB.

Pemberlakuan UU Nomor 12 Tahun 1985 tentang Pajak Bumi dan Bangunan didasari pemikiran antara lain bahwa bumi dan bangunan memberikan keuntungan dan atau kedudukan sosial ekonomi yang lebih baik bagi orang atau badan yang mempunyai suatu hak atasnya dan memperoleh manfaat darinya, oleh sebab itu wajar apabila kepada mereka diwajibkan memberikan sebagian dari manfaat atau kenikmatan yang diperolehnya kepada negara melalui pajak.

Pelaksanaan reformasi di bidang pajak atas tanah dan bangunan disamping berupaya menyederhanakan berbagai pungutan pajak atas tanah dan bangunan juga tetap memberikan tekanan terhadap upaya untuk meningkatkan penerimaan dan memperhatikan aspek keadilan serta meminimalkan dampak terhadap distorsi kegiatan ekonomi dan sosial mengingat PBB merupakan salah satu sumber utama penerimaan daerah mengingat PBB adalah penerimaan pajak Pusat yang keseluruhan hasilnya diserahkan kepada Daerah.

Menyadari pentingnya penerimaan PBB bagi pembiayaan pembangunan Daerah, maka pada tahun 1989 dilakukan pembaharuan sistem administrasi penerimaan PBB melalui Sistem Tempat Pembayaran (SISTEP). SISTEP diujicobakan pertama kali pada tahun 1989 di wilayah Kabupaten Tangerang, yang secara bertahap sehingga pada akhir tahun 1994 seluruh Kabupaten/Kota di Indonesia telah melaksanakan sistem administrasi pemungutan PBB dengan pola SISTEP.

Pokok-pokok ketentuan SISTEP antara lain meliputi :

\begin{itemize}
  \item Hanya ada satu tempat pembayaran untuk setiap wilayah pembayaran PBB tertentu sebagaimana tercantum dalam Surat Pemberitahuan Pajak Terhutang (SPPT) yang diusahakan berdekatan dengan lokasi objek pajak.
  \item Pembayaran PBB dilakukan sekaligus dalam satu kali pembayaran dan tidak dapat diangsur.
  \item Jatuh tempo pembayaran PBB diatur seragam sehingga hanya terdapat satu tanggal jatuh tempo.
  \item Surat Tanda Terima Setoran (STTS) PBB telah tersedia di tempat pembayaran sebelum SPPT diterima oleh wajib pajak.
  \item Administrasi PBB harus dilaksanakan dengan dukungan komputer.
  \item Sistem Pemantauan STTS dan pelaporan pembayaran didesain sedemikian rupa sehingga perkembangan pembayaran PBB diketahui lebih cepat oleh instansi terkait.
  \item Secara sistem, SISTEP mampu menerbitkan daftar negatif (\textit{negative list}) wajib pajak yang tidak memenuhi kewajiban PBB pada saat jatuh tempo pembayaran sehingga penegakan hukum dapat dilaksanakan.
\end{itemize}

Dengan diberlakukannya SISTEP, penerimaan PBB mengalami peningkatan yang berarti. Keberhasilan pembaharuan sistem administrasi pemungutan dengan pola SISTEP terutama menyangkut perubahan sistem pemungutan PBB yang sebelumnya dilakukan oleh petugas pemungut desa/kelurahan secara bertahap diambil alih melalui sistem perbankan yang ditunjang dengan komputerisasi administrasi penerimaan PBB. Pemberian pelayanan kepada wajib pajak dalam rangka pembayaran PBB menjadi lebih mudah dan pelaksanaan penegakan hukum dapat lebih ditingkatkan.

Pada tahun 1994 Pemerintah menerbitkan UU Nomor 12 Tahun 1994 tentang Perubahan UU Nomor 12 Tahun 1985 tentang Pajak Bumi dan Bangunan, esensi dari perubahan ini adalah penyesuaian atas praktek perkembangan penyelenggaraan kegiatan usaha yang belum tertampung dalam UU Nomor 12 Tahun 1985. Pada saat inilah Sistem Manajemen Informasi Objek Pajak yang dikenal dengan nama SISMIOP muncul dengan perbaikan pada manajemen objek pajak berupa pemberian Nomor Objek Pajak (NOP) yang baku.

Selanjutnya seiring dengan bergulirnya reformasi di Indonesia, dan sejalan dengan tuntutan otonomi daerah setelah masa reformasi maka beberapa Pemda menuntut agar PBB P2 (Perdesaan dan Perkotaan) menjadi pajak daerah, namun tidak semua Pemda setuju sehingga terjadi perdebatan pro dan kontra pengalihan PBB P2 menjadi pajak daerah. Pemda yang setuju adalah mereka yang memiliki potensi penerimaan PBB P2 yang besar (kota-kota besar di Indonesia) karena akan menambah jumlah PAD-nya, jika masih menjadi pajak pusat maka PBB P2 yang kembali ke daerah penghasil dalam bentuk Dana Bagi Hasil (DBH) hanya 64,8\% saja. Sementara itu Pemda yang tidak setuju dengan peralihan PBB P2 menjadi pajak daerah adalah Pemda yang memiliki potensi penerimaan PBB P2 yang kecil (Daerah yang bukan kota besar di Indonesia), selain itu dikeluhkan juga faktor lain diluar pendapatan seperti :

\begin{itemize}
  \item Sarana dan prasarana yang mendukung
  \item Keterbatasan SDM yang mampu mengelola PBB P2
  \item Regulasi dan SOP Pelayanan
  \item Sistem Manajemen Informasi Objek Pajak
\end{itemize}

Selain itu tantangan pengalihan PBB P2 bagi Pemda antara lain kesiapan Kabupaten/Kota pada masa awal pengalihan yang belum optimal, sehingga dapat berdampak pada penurunan pelayanan, penerimaan, dan beberapa hal seperti :

\begin{itemize}
  \item Kesenjangan (disparitas) kebijakan PBB P2 antar Kabupaten/Kota.
  \item Hilangnya potensi penerimaan bagi Provinsi (16,2\%) dan hilangnya potensi penerimaan insentif PBB khususnya bagi Kabupaten/Kota yang potensi PBB P2-nya rendah.
  \item Beban biaya pemungutan PBB-P2 yang cukup besar.
\end{itemize}

Kemudian pada tanggal 15 September 2009 terbitlah UU Nomor 28 Tahun 2009 tentang Pajak Daerah dan Retribusi Daerah dimana telah mengakomodir PBB P2 menjadi pajak daerah yang paling lambat dilaksanakan pada tanggal 1 Januari 2014.

Adapun tujuan dari pengalihan PBB-P2 menjadi pajak daerah adalah untuk meningkatkan \textit{local taxing power} pada Kabupaten/Kota seperti : 

\begin{itemize}
  \item Memperluas objek pajak daerah dan retribusi daerah
  \item Menambah jenis pajak daerah dan retribusi daerah (termasuk pengalihan PBB Perdesaan dan Perkotaan di Pajak Daerah)
  \item Memberikan diskresi penetapan tarif pajak kepada daerah
  \item Menyerahkan fungsi pajak sebagai instrumen penganggaran dan pengaturan pada daerah.
\end{itemize}

Pada awal pengalihannya, kondisi pembayaran PBB P2 di Kabupaten Brebes menggunakan sistem \textit{single host} dimana data ditempatkan dalam 2 (dua) tempat, yaitu di Dinas Pendapatan dan Pengelolaan (DPPK) sebagai Dinas yang mengelola PBB P2, dan Bank Pembangunan Daerah (BPD) Jawa Tengah sebagai tempat pembayaran. 

Data akan selalu diselaraskan dalam periode harian, sehingga data pembayaran yang masuk hari ini akan tercatat dalam basis data DPPK pada H+1, begitupun segala perubahan yang terjadi di basis data DPPK pada hari ini akan dicatatkan perubahannya pada basis data BPD Jawa Tengah di H+1.

Kegiatan ini menjadikan data tidak konsisten dan terjadi perbedaan data diantara 2 (dua) basis data.

\subsection{Peluang Yang Dituju}

Dari kondisi data yang tidak konsisten seperti dijelaskan diatas. Maka diharapkan dengan membangun sistem Host-to-Host atau dengan kata lain seluruh data berada di DPPK sebagai pengelola PBB P2, sedangkan tempat pembayaran akan meminta data sesuai dengan objek yang akan dibayarkan saja.
                              
\section{PEMBENTUKAN SASARAN SISTEM BARU SECARA KESELURUHAN}




\section{PENGIDENTIFIKASIAN PARA PEMAKAI SISTEM}


\section{PEMBENTUKAN LINGKUP SISTEM}


\section{PENGUSULAN PERANGKAT LUNAK DAN PERANGKAT KERAS UNTUK SISTEM BARU}


\section{PEMBUATAN ANALISIS UNTUK MEMBANGUN SENDIRI ATAU MEMBELI APLIKASI}


\section{PEMBUATAN ANALISIS BIAYA / MANFAAT}


\section{PENGKAJIAN TERHADAP RESIKO PROYEK}


\section{PEMBERIAN REKOMENDASI UNTUK MENERUSKAN ATAU MENGHENTIKAN PROYEK}


\end{document}  